\documentclass[10pt]{article}

\usepackage[letterpaper, margin=.7in]{geometry}
\usepackage{multicol}
\usepackage{hyperref}
\usepackage{indentfirst}
\usepackage{float}
\usepackage{graphicx}
\usepackage{pdfpages}
\usepackage[english]{babel}
\usepackage[autostyle, english=american]{csquotes}

\graphicspath{{G:/Simulation/Development/Gauges/HornetFCS/Resources/Python/}}
\setlength\columnsep{0.3125in}

\title{Flight Control Systems for Microsoft Flight Simulator X}
\author{Orion Lyau}
\date{\today}

\begin{document}
\begin{multicols}{2}
[
\maketitle
\begin{abstract}
\noindent As a desktop flight simulator intended for personal entertainment use, Microsoft Flight Simulator X has limited capability in simulating complex fly-by-wire flight control systems found in modern aircraft.  This lack of functionality is particularly apparent in aircraft such as the Airbus A320 or Boeing F/A-18 Hornet, where flight computers provide unique flight characteristics and thus direct linkage of control surfaces to flight controls is undesirable.  This paper aims to describe methodology used to emulate rudimentary fly-by-wire functionality within the scope of Flight Simulator X.
\end{abstract}
]

\section{Introduction}

In fly-by-wire systems, pilot inputs are not directly linked to aircraft control surfaces; instead, pilot inputs are sent to a flight computer, which then determines how to deflect the control surfaces in order to achieve the desired response.  By utilizing fly-by-wire, control laws can be implemented to help increase an aircraft's stability as well as implement flight envelope protections to help prevent unsafe operations.

As the Airbus A321 is an aircraft included by default with Flight Simulator X, the simulator has limited fly-by-wire capability built in.  Unfortunately, this functionality is poorly documented and is limited to a single on/off parameter in the aircraft.cfg file and a set of key events that simply toggle whether the flight computer is on or off, with no apparent method of customization.  Although the 2007 release of the Acceleration expansion pack included the Boeing F/A-18 Hornet, no additions were made in regards to fly-by-wire functionality.

However, Flight Simulator X was the first simulator in the Microsoft Flight Simulator series to feature an API that allows software add-on developers to create add-ons that interface directly with the simulator.  This API, called SimConnect, allows the masking of simulator events, including those produced by the movement of flight control peripherals.  The SDK states that \textquote{it is possible to mask ESP events, and therefore intercept them before they reach the simulation engine, and perhaps send new events to the simulation engine after appropriate processing has been done.}  By masking events generated from user input, it is possible to do further processing to create custom fly-by-wire capability.

On a high level, fly-by-wire is a simple concept --- the idea that pilot inputs are processed using a computer before control surfaces are deflected accordingly is fairly straightforward ---  however, its implementation in actual aircraft is quite complex.  As Flight Simulator X is intended as an personal entertainment product (and because the author of this paper is currently an undergrad student not majoring in aerospace engineering), the discussion of a simulated flight control system in this paper will be vastly simplified in comparison to flight control systems used in real aircraft.  It is beyond the scope of this paper to provide an in-depth and technical discussion of topics important to real flight control systems, such as signal processing, control theory, and aircraft dynamics.

\section{Implementation}

A proof of concept fly-by-wire flight control system for Flight Simulator X would be the trivial case where a SimConnect client masks the pilot's input, then sends new events with the same data to the simulator.  While this would be a very boring fly-by-wire system --- essentially an inefficient way of doing what Flight Simulator already does on its own --- it demonstrates the concept of masking events and sending new data generated by an external SimConnect client.


%linewidth: \the\linewidth

\begin{figure}[H]
\centering
\includegraphics[scale=1.0, page=1]{plot.pdf}
\caption{G-force vs stick position}
\end{figure}

%\includepdf[pages=1]{plot.pdf}

\end{multicols}
\end{document}